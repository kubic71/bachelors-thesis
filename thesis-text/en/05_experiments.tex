\chapter{Experiments}
\label{experiments_chap}

\section{Blackbox PoC}
Here we go into more technical details about previously mentioned blackbox attacks we initially tried.

\begin{itemize}
    \item TREMBA
    \item RayS
    \item SquareAttack
    \item Sparse-RS
\end{itemize}

\subsection{TREMBA}

\subsection{RayS}
This one is hard label attack and doesn't use the continuos loss from GVision.

\subsection{SquareAttack}

\subsubsection{SquareAttack L2}
Show an image of nice cat.

\subsubsection{SquareAttack Linf}

\subsubsection{Comparison with local success-rate}



\section{Local transferability experiments}
Here go all the different transfer-matrices

\subsection{Choice of dataset}

\subsection{Choice of local models}
We performed all our experiments on the following pretrained PyTorch ImageNet models.

\begin{itemize}
    \item ResNet-18, ResNet-50 (\cite{he2015deep})
    \item ResNeXt-50 (32x4d) (\cite{xie2017aggregated})
    \item Wide-ResNet-50-2 (\cite{zagoruyko2017wide})
    \item Squeezenet (\cite{iandola2016squeezenet})
    \item DenseNet-121 (\cite{huang2018densely})
    \item EfficientNet-b0 (\cite{tan2020efficientnet})
    \item EfficientNet-b0 adversarially trained (\cite{tramer2020ensemble})
\end{itemize}


Apart from EfficientNets, all models were taken from the \href{https://pytorch.org/vision/stable/models.html}{torchvision.models} Python package. For the EfficientNets we used \href{https://github.com/lukemelas/EfficientNet-PyTorch}{github.com/lukemelas/EfficientNet-PyTorch} reimplementation, because the original implementation uses in Tensorflow.

\subsection{Baseline}
\subsection{Baseline}
\subsection{Baseline}
\subsection{Baseline}

\section{Transferability evaluation on GVision}
TODO: just run the images against GVision

\subsection{Choice of evaluation metrics}

\subsection{Wordcloud}